\documentclass[8pt]{beamer}

\usepackage{cmap}
\usepackage[T2A]{fontenc}
\usepackage[utf8]{inputenc}
\usepackage{amsfonts}
\usepackage{amsthm}
\usepackage{mathtools}
\usepackage{color}
\usepackage{hyperref}
\usepackage{graphicx}
\usepackage{pdfpages}
\usepackage{forest}
\usepackage{adjustbox}
\usepackage{times}
\usepackage{tikz}

\mode<presentation>{
    \usetheme{Marburg}
    \usecolortheme{sidebartab}
}

% \newtheorem{theorem}{Theorem}[section]
% \newtheorem{lemma}{Lemma}[section]
% \newtheorem{proposition}{Proposition}[section]
% \newtheorem{corollary}{Corollary}[section]
% \newtheorem{definition}{Definition}[section]
% \newtheorem{remark}{Remark}[section]
% \newtheorem{example}{Example}[section]

\newcommand{\E}{\ensuremath{\mathbb{E}}}
\newcommand{\D}{\ensuremath{\mathbb{D}}}
\renewcommand{\C}{\ensuremath{\mathbb{C}}}
\newcommand{\R}{\ensuremath{\mathbb{R}}}
\newcommand{\Q}{\ensuremath{\mathbb{Q}}}
\renewcommand{\|}{\ensuremath{\hspace{0.1cm} | \hspace{0.1cm}}}

\newcommand{\red}[1]{\textcolor{red}{#1}}
\newcommand{\blue}[1]{\textcolor{blue}{#1}}
\newcommand{\green}[1]{\textcolor{green}{#1}}
\newcommand{\orange}[1]{\textcolor{orange}{#1}}
\newcommand{\teal}[1]{\textcolor{teal}{#1}}
\newcommand{\purple}[1]{\textcolor{purple}{#1}}

\renewcommand{\phi}{\varphi}
\renewcommand{\epsilon}{\varepsilon}
\renewcommand{\le}{\leqslant}
\renewcommand{\ge}{\geqslant}


\begin{document}
    \title[Liouville's Theorem]{Liouville's Theorem (Differential algebra)}
    \author{Vanya Vorobiov}
    \date{\today}
    \institute{Sber}

    \begin{frame}
        \titlepage
    \end{frame}

    \section{Basic definitions}
    \begin{frame}{Basic definitions}
        \begin{block}{Remark}<1->
            Through the all of presentation we will suppose that all fields have 0 characteristic.
        \end{block}
        \begin{definition}<2->
            Field $F$ is differential if it's equipped with the unary function $'$ such that:
            \begin{itemize}
                \item $(a+b)' = a' + b'$
                \item $(ab)' = a'b + ab'$
            \end{itemize}
        \end{definition}
        \begin{definition}<3->
            Subfield $K \subseteq F$, $K = \{ a \in F \| a' = 0 \}$ is called subfield of constants.
        \end{definition}
        \begin{definition}<4->
            Differential extension of the differential field $F$ is field $E$ such that $E \supseteq F$ and there is the same differentiation $'$ on $E$.
        \end{definition}
        \begin{definition}<5->
            Let $F$ be the differential field. Then
            \begin{itemize}
                \item $b$ is called the logarithm of $a$ if $b' = \frac{a'}{a}$
                \item $b$ is called the exponent of $a$ if $a' = \frac{b'}{b}$
            \end{itemize}
        \end{definition}
    \end{frame}
    
    

\section{What is integrability in elementary functions}
\begin{frame}{What is integrability in elementary functions}
    \begin{definition}<1->
        The extension $E$ of $F$ is called elementary if it can be presented as $E = F(t_1, \ldots, t_n)$ and for all $i$ $t_i$ is logarithm or exponent or algebraic over $F(t_1, \ldots, t_{i-1})$.
    \end{definition}
    \begin{block}{Remark}<2->
        Common sense say us that any function $f: \C \to \C$ is elementary iff it can be constucted via finite number of radicals, sines, cosines, exponents, logarithms and hyperbolic functions.
        One can see that it's consistent with our approach.
    \end{block}
\end{frame}


\end{document}
