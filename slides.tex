\documentclass[8pt]{beamer}

\usepackage{cmap}
\usepackage[T2A]{fontenc}
\usepackage[utf8]{inputenc}
\usepackage{amsfonts}
\usepackage{amsthm}
\usepackage{mathtools}
\usepackage{color}
\usepackage{hyperref}
\usepackage{graphicx}
\usepackage{pdfpages}
\usepackage{forest}
\usepackage{adjustbox}
\usepackage{times}
\usepackage{tikz}

\mode<presentation>{
    \usetheme{Marburg}
    \usecolortheme{sidebartab}
}

% \newtheorem{theorem}{Theorem}[section]
% \newtheorem{lemma}{Lemma}[section]
% \newtheorem{proposition}{Proposition}[section]
% \newtheorem{corollary}{Corollary}[section]
% \newtheorem{definition}{Definition}[section]
% \newtheorem{remark}{Remark}[section]
% \newtheorem{example}{Example}[section]

\newcommand{\E}{\ensuremath{\mathbb{E}}}
\newcommand{\D}{\ensuremath{\mathbb{D}}}
\renewcommand{\C}{\ensuremath{\mathbb{C}}}
\newcommand{\R}{\ensuremath{\mathbb{R}}}
\newcommand{\Q}{\ensuremath{\mathbb{Q}}}
\renewcommand{\|}{\ensuremath{\hspace{0.1cm} | \hspace{0.1cm}}}

\newcommand{\red}[1]{\textcolor{red}{#1}}
\newcommand{\blue}[1]{\textcolor{blue}{#1}}
\newcommand{\green}[1]{\textcolor{green}{#1}}
\newcommand{\orange}[1]{\textcolor{orange}{#1}}
\newcommand{\teal}[1]{\textcolor{teal}{#1}}
\newcommand{\purple}[1]{\textcolor{purple}{#1}}

\renewcommand{\phi}{\varphi}
\renewcommand{\epsilon}{\varepsilon}
\renewcommand{\le}{\leqslant}
\renewcommand{\ge}{\geqslant}


\begin{document}
    \title[Liouville's Theorem]{Liouville's Theorem \\ on integrability via elementary functions}
    \author{Vanya Vorobiov}
    \date{\today}
    \institute{Sber}

    \begin{frame}
        \titlepage
    \end{frame}

    \section{Introduction}
    \begin{frame}{Introduction}
        \uncover<1->{
            From high school, we are familiar with the idea that some integrals cannot be expressed in terms of elementary functions. For instance:
            \[
                \int e^{\pm x^2} \, dx, \quad \int \frac{dx}{\ln x}, \quad \int \frac{e^x}{x} \, dx, \quad \int \frac{\sin x}{x} \, dx, \quad \int \frac{\sinh x}{x} \, dx, \quad \int \log \log x \, dx
            \]

            These examples highlight the limitations of elementary functions in representing certain integrals.
        }
        
        \uncover<2->{
            In this presentation, we will:
            \begin{itemize}
                \item Introduce and prove a powerful tool: \textbf{Liouville's theorem}.
                \item Derive these integrals as a consequence.
                \item If time permits, discuss some special from of \textbf{elliptic integrals}, which also cannot be expressed in elementary terms.
                \[
                \int \frac{dx}{\sqrt{P(x)}}
                \]
                for $\deg P = 2, 3$ and $P$ hasn't multiple roots.
            \end{itemize}
        }
    \end{frame}

    \section{Basic definitions}
    \begin{frame}{Basic definitions}
        \begin{block}{Remark}<1->
            Through the all of presentation we will suppose that all fields have 0 characteristic.
        \end{block}
        \begin{definition}<2->
            Field $F$ is differential if it's equipped with the unary function $'$ such that:
            \begin{itemize}
                \item $(a+b)' = a' + b'$
                \item $(ab)' = a'b + ab'$
            \end{itemize}
        \end{definition}
        \begin{definition}<3->
            Subfield $K \subseteq F$, $K = \{ a \in F \| a' = 0 \}$ is called subfield of constants.
        \end{definition}
        \begin{definition}<4->
            Differential extension of the differential field $F$ is field $E$ such that $E \supseteq F$ and there is the same differentiation $'$ on $E$.
        \end{definition}
        \begin{definition}<5->
            Let $F$ be the differential field. Then
            \begin{itemize}
                \item $b$ is called the logarithm of $a$ if $b' = \frac{a'}{a}$
                \item $b$ is called the exponent of $a$ if $a' = \frac{b'}{b}$
            \end{itemize}
        \end{definition}
    \end{frame}
    
    

    \section{What is expression in elementary terms}
    \begin{frame}{What is expression in elementary terms}
        \begin{definition}<1->
            The extension $E$ of $F$ is called elementary if it can be presented as $E = F(t_1, \ldots, t_n)$ and for all $i$ $t_i$ is logarithm or exponent or algebraic over $F(t_1, \ldots, t_{i-1})$.
        \end{definition}
        \begin{block}{Remark}<2->
            Common sense says us that some function $f: \C \to \C$ is elementary iff it can be constucted via finite number of radicals, sines, cosines, exponents, logarithms and hyperbolic functions.
            One can see that it's consistent with our approach. Futhermore our definition on elementarity is more general.
        \end{block}
    \end{frame}

    \begin{section}{Liouville's Theorem (statement)}
    \begin{frame}{Liouville's Theorem (statement)}
        \begin{theorem}[Liouville, 1833-1841]
            Let $F$ be a differential field, and $K$ is its subfield of constants. 
            If for $\alpha\in F$ equation $x' = \alpha$ has the solution in some elementary extension of $F$, such that its subfield of constants is still $K$, then
            \[
                \alpha = \sum\limits_{i=1}^m c_i \frac{u_i'}{u_i} + v'
            \]
            for some $c_1, \ldots c_m \in K$, $u_1, \ldots, u_m, v \in F$.
        \end{theorem}
    \end{frame}

    \section{The Main Lemma}
    \begin{frame}{The Main Lemma}
        \begin{lemma}<1->
            Let $F$ be a differential field, $t$ is trancendental over $F$, and $t$ is a logarithm or an exponent of some element from $F$. 
            And let $f\in F[x]$ be a polynomial, $\deg{f} = k \ge 1$
            \begin{itemize}
                \item If $t$ is a logarithm then the degree of $(f(t))'$ is $k$ if the leading coefficient of is not a constant, and it has degree $k-1$ if  the leading coefficient is a constant.
                \item If $t$ is an exponent then the degree of $(f(t))'$ is $k$ and it's multiple of $f$ if and only if $f$ is a monomial.
            \end{itemize}
        \end{lemma}

        \begin{proof}<2->
            It's a quite simple technical exercise.
        \end{proof}
    \end{frame}

    \section{Liouville's Theorem (proof)}
    \begin{frame}{Liouville's Theorem (proof)}
        \uncover<1->{
            Let $x$ be the solution of differential equation mentioned above. And $x\in F(t_1, \ldots, t_n)$.

            We will use induction on $n$ (we don't fix the field $F$).

            For short we denote $t = t_1$.
        }

        \uncover<2->{
            Using the inductive assumption, we get
            \[
                \alpha = \sum\limits_{i=1}^m c_i \frac{u_i'}{u_i} + v'
            \]
            for some $c_1, \ldots c_m \in K$, $u_1, \ldots, u_m, v \in F(t)$.
            
            Here we use that the subfield of constants of $F(t)$ is $K$.
        }

        \uncover<3->{
            Now we consider 3 cases 
            \begin{itemize}
                \item $t$ is trancendental over $F$ and it is a logarithm;
                \item $t$ is trancendental over $F$ and it is an exponent;
                \item $t$ is algebraic over $F$.
            \end{itemize}
        }
    \end{frame}

    \subsection{$t$ is a trancendental logarithm}
    \begin{frame}{$t$ is a trancendental logarithm}
        \uncover<1->{
            Firstly let us consider the basic properties of logarithmical derivatives:
            \[
            \frac{(ab)'}{ab} = \frac{a'}{a} + \frac{b'}{b}, \quad \frac{(1/a)'}{1/a} = - \frac{a'}{a}
            \]
        }
        
        \uncover<2->{
            Then we can suppose that all of $u_i$ are distinct monic irreducible polynomials.
        }

        \uncover<3->{
            Now suppose that some $u_i \notin F$. It's clear that $\frac{u_i'}{u_i}$ is already in lowest terms (because $\deg u_i > \deg u_i'$).
        }

        
        \begin{itemize}
            \uncover<4->{
                \item If there's not $u_i$ in the denominator of $v$, then there's not $u_i$ in the denominator of $v'$. But then $\alpha \notin F$.
            }
            \uncover<5->{
                \item If there's $u_i$ in the denominator of $v$, then the demominator of $v'$ is divisible by $u_i^2$ and it still cannot be reduced in the general sum and $\alpha \notin F$.
            }
        \end{itemize}

        \uncover<6->{
            Therefore $u_1, \ldots, u_m \in F$ and $v' \in F$
        }

        \uncover<7->{
            Then 
            \[
                v' = (ct + s)' = ct' + s' = c \frac{z'}{z} + s'
            \]
            \[
                \alpha = \sum c_i \frac{u_i'}{u_i} + c \frac{z'}{z} + s'
            \]
        QED.
        }
        

    \end{frame}

    \subsection{$t$ is a trancendental exponent}
    \begin{frame}{$t$ is a trancendental exponent}
        \uncover<1->{
            Let $u_1, \ldots, u_m$ be distinct monic irreducible again.
        }
        \uncover<2->{
            It's clear that $u_i = t$ is only $u$ that can be not in $F$ because only irreducible monomial, we would get the same contradiction otherwise.
        }
        \uncover<3->{
            By the main lemma $v'\in F$ iff $v\in F$.
        }
        \unceover<4>{
            Then 
            \[
                \alpha = c_1 \frac{t_1'}{t_1} + \sum_\limits_{i=2} \frac{u_i'}{u_i}^m + v' = \sum_\limits_{i=2}^m \frac{u_i'}{u_i} + (v + c_1 z)'
            \]
        }

    \end{frame}

    \subsection{$t$ is algebraic}
    \begin{frame}{$t$ is algebraic}
        TODO
    \end{frame}

    \section{Corollaries}
    \subsection{The main corollary}
    \begin{frame}{The main corollary}
        TODO
    \end{frame}
    
    \subsection{Some special cases}
    \begin{frame}
        TODO
    \end{frame}

    \subsection{On elliptic integrals}
    \begin{frame}{On elliptic integrals}
        TODO
    \end{frame}


\end{document}
